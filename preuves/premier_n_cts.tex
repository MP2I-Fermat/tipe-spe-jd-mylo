\documentclass[a4paper, 12pt]{article}
\usepackage{amssymb}
\usepackage{amsmath}

\title{Sur les variations des suites $premier_n(\overrightarrow{\alpha})$}
\author{Mylo FAWCETT}

\begin{document}
\maketitle

Puisque notre grammaire ne contient pas de re\`gles $N \rightarrow \varepsilon$,
on peut de\'finir les suites
$(premier_i(\overrightarrow{\alpha}))_{i \in \mathbb{N}}$ comme suit:

\begin{center}
    \begin{tabular}{ l l }
        $premier_0(a) = \{a\}$                           & \text{lorsque $a$ est un terminal} \\
        $premier_0(\overrightarrow{\alpha}) = \emptyset$ & \text{sinon}                       \\
    \end{tabular} \\
    \bigskip
    $\begin{aligned}
             & \begin{aligned}
                   premier_{n+1}(N) = & \thinspace premier_n(N) \bigcup_{N \rightarrow \overrightarrow{\alpha}} premier_n(\overrightarrow{\alpha}) \\
                                      & \thinspace \text{si $N$ est un non-terminal}
               \end{aligned} \\
             & premier_{n+1}(\alpha_1...\alpha_i) = premier_n(\alpha_1...\alpha_i) \cup premier_n(\alpha_1) \thickspace \text{sinon}
        \end{aligned}$

\end{center}

On d\'emontre alors les propri\'et\'es suivantes:

\bigskip

\textit{Propri\'et\'e 1}:
Si pour toute r\`egle $N \rightarrow \overrightarrow{\alpha}$,
$(premier_i(\overrightarrow{\alpha}))$ est constant a partir du rang $n$ et
$premier_n(\overrightarrow{\alpha}) \subset premier_n(N)$,
alors $(premier_i(N))$ est constant a partir du rang $n$.

\textit{D\'emonstration}:
Par r\'ecurrence sur $k \in \mathbb{N}$ avec $P(k): premier_{n+k}(N) = premier_n(N)$:
\begin{itemize}
    \item $P(0)$ est imm\'ediat.
    \item Soit $k \in \mathbb{N}$ et supposons $P(k)$ vrai. Alors:
          \par $\begin{aligned}
                  premier_{n+k+1}(N) = & \thinspace premier_{n+k}(N) \bigcup_{N \rightarrow \overrightarrow{\alpha}} premier_{n+k}(\overrightarrow{\alpha}) \\
                  =                    & \thinspace premier_{n+k}(N) \bigcup_{N \rightarrow \overrightarrow{\alpha}} premier_n(\overrightarrow{\alpha})     \\
                                       & \thinspace \text{\small (car $(premier_i(\overrightarrow{\alpha}))$ est constant a partir du rang $n$)}            \\
                  =                    & \thinspace premier_n(N) \bigcup_{N \rightarrow \overrightarrow{\alpha}} premier_n(\overrightarrow{\alpha})         \\
                                       & \thinspace \text{\small (d'\`apres $P(k)$)}                                                                        \\
                  =                    & \thinspace premier_n(N)                                                                                            \\
                                       & \thinspace\text{\small (car $premier_n(\overrightarrow{\alpha}) \subset premier_n(N)$)}
              \end{aligned}$
          \par D'o\`u $P(k+1)$.
\end{itemize}

\bigskip

\textit{Propri\'et\'e 2}:
Soit $\overrightarrow{\alpha} = \alpha_1...\alpha_m$ une d\'erivation avec $m > 1$ ou $\alpha_1$ terminal. Si
$(premier_i(\alpha_1))$ est constant \`a partir du rang $n$ et
$permier_n(\alpha_1) \subset premier_n(\overrightarrow{\alpha})$, alors
$(permier_i(\overrightarrow{\alpha}))$ est constant \`a partir du range $n$.

\textit{D\'emonstration}:
Par r\'ecurrence sur $k \in \mathbb{N}$ avec $P(k): premier_{n+k}(\overrightarrow{\alpha}) = premier_n(\overrightarrow{\alpha})$

\begin{itemize}
    \item $P(0)$ est imm\'ediat.
    \item Soit $k \in \mathbb{N}$ et supposons $P(k)$ vrai. Alors:
          \par $\begin{aligned}
                  premier_{n+k+1} = & \thinspace premier_{n+k}(\overrightarrow{\alpha}) \cup premier_{n+k}(\alpha_1)                 \\
                  =                 & \thinspace premier_{n+k}(\overrightarrow{\alpha}) \cup premier_n(\alpha_1)                     \\
                                    & \thinspace \text{\small (car $(premier_i(\alpha_1))$ est constant \`a partir du rang $n$)}     \\
                  =                 & \thinspace premier_n(\overrightarrow{\alpha}) \cup premier_n(\alpha_1)                         \\
                                    & \thinspace \text{\small (d'\`apres $P(k)$)}                                                    \\
                  =                 & \thinspace premier_n(\overrightarrow{\alpha})                                                  \\
                                    & \thinspace\text{\small (car $premier_n(\alpha_1) \subset premier_n(\overrightarrow{\alpha})$)}
              \end{aligned}$ \\
          D'o\`u $P(k+1)$
\end{itemize}


\end{document}